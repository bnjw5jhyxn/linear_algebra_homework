\def\zero{{\bf 0}}
\def\real{{\bf R}}
\centerline{Steve Hsu\hfill homework 1}
\medskip\goodbreak
\item{1.2.9.} Corollary 1 of Theorem 1.1

We want to show that $\zero$ is unique.
In other words, if $\zero$ and $\zero'$ are both additive identities
in a vector space $V$, then they are equal.

Suppose that a $V$ has two additive identities, $\zero$ and $\zero'$.
We want to show that $\zero = \zero'$.
First, let $x \in V$ be an arbitrary member of the vector space.
Since $\zero$ and $\zero'$ are additive identities,
$x + \zero = x = x + \zero'$.
Addition is commutative by (VS 1), so $\zero + x = \zero' + x$.
By the cancellation theorem, we can cancel $x$ from both sides,
giving $\zero = \zero'$, as desired.

\item{} Corollary 2 of Theorem 1.1

We want to show that each element of $V$ has exactly one additive inverse.
In other words, if $y$ and $y'$ are additive inverses of some $x \in V$,
then $y = y'$.

Let $x \in V$ be arbitrary.
Let $y, y' \in V$ be additive inverses of $x$.
We know that these exist by (VS 4).
Since $y$ and $y'$ are additive inverses of $x$,
$x + y = \zero = x + y'$,
Addition is commutative by (VS 1), so $y + x = y' + x$.
By the cancellation theorem, we can cancel $x$ from both sides,
giving $y = y'$, as desired.

\item{} Theorem 1.2(c)

We want to show that $a\zero = \zero$ for all $a \in F$.

Let $a \in F$ be arbitrary.
By (VS 7), $a\zero + a\zero = a(\zero + \zero)$.
Since $\zero$ is the additive identity by (VS 3),
$a(\zero + \zero) = a\zero = a\zero + \zero$.
Since addition is commutative by (VS 1), $a\zero + \zero = \zero + a\zero$.
By applying the cancellation theorem to $a\zero + a\zero = \zero + a\zero$,
we have $a\zero = \zero$, as desired.
\medskip\goodbreak
\item{1.2.13.}

$V$ is not a vector space over $\real$
because it does not satisfy (VS 8).

For example, $(1 + 1)(1,2) = 2(1,2) = (2,2)$,
but $1(1,2) + 1(1,2) = (1,2) + (1,2) = (2,4)$.
\medskip\goodbreak
\item{1.3.8.} a.

By Theorem 1.3, it suffices to show that
$W_1$ contains $\zero$, it is closed under addition,
and it is closed under scalar multiplication.

Notice that a vector $a \in \real^3$ is in $W_1$ if and only if
it can be written in the form $x(1,3,-1)$ for some $x \in \real$.
It is now clear that $\zero = 0(1,3,-1) \in W_1$.
$W_1$ is closed under addition since
$x(1,3,-1) + y(1,3,-1) = (x + y)(1,3,-1)$ by (VS 8) in $\real^3$
and $W_1$ is closed under scalar multiplication since
$(c)x(1,3,-1) = (cx)(1,3,-1)$ by (VS 6) in $\real^3$.

\item{} e.

$W_5$ does not contain $\zero$ since $0 + 2(0) - 3(0) \ne 1$.
\medskip\goodbreak
\item{1.3.18.} forward direction

Assume that $W \subseteq V$ is a subspace of $V$.
We want to show that $\zero \in W$ and
$ax + y \in W$ for all $a \in F$ and $x,y \in W$.

By Theorem 1.3, since $W$ is a subspace of $V$,
$W$ contains $\zero$ and is closed under addition and scalar multiplication.

$\zero \in W$ by Theorem 1.3.

Let $a \in F$ and $x,y \in W$ be arbitrary.
Since $W$ is closed under scalar multiplication, $ax \in W$.
Since $W$ is closed under addition, $ax + y \in W$, as desired.

\item{} backward direction

Assume that $W \subseteq V$ contains $\zero$ and that
$ax + y \in W$ for all $a \in F$ and $x,y \in W$.
We want to show that $W$ is a subspace of $V$.

By Theorem 1.3, we need to show that $\zero \in W$
and that $W$ is closed under addition and scalar multiplication.

We assumed that $\zero \in W$.

Let $x,y \in W$ be arbitrary.
We want to show that $x + y \in W$.

By our assumption, $1x + y \in W$.
By (VS 5) in $V$, $1x = x$, giving $x + y \in W$, as desired.

Let $a \in F$ and $x \in W$ be arbitrary.
We want to show that $ax \in W$.

By our assumption, $ax + \zero \in W$.
Since $ax + \zero = ax$ by (VS 3) in $V$,
$ax \in W$, as desired.
\medskip\goodbreak
\item{1.3.19.} forward direction

Assume that $W_1,W_2 \subseteq V$ are subspaces of $V$
such that $W_1 \cup W_2$ is also a subspace of $V$.
We want to show that $W_1 \subseteq W_2$ or $W_2 \subseteq W_1$.

Assume for contradiction that
$W_1 \not\subseteq W_2$ and $W_2 \not\subseteq W_1$.
Then there is an $x \in W_1$ that is not in $W_2$
and a $y \in W_2$ that is not in $W_1$.

Since $W_1 \cup W_2$ is a subspace of $V$, $x + y \in W_1 \cup W_2$.
We now have two cases: first, $x + y \in W_1$; and second, $x + y \in W_2$.

In the first case, since $W_1$ is a subspace of $V$
and $x \in W_1$, $x$ has an additive inverse $u$ in $W_1$.
Then $x + y + u \in W_1$ since $W_1$ is a subspace of $V$.
This a contradiction because $x + y + u = y + x + u = y + \zero = y$,
which we assumed is not in $W_1$.

In the second case, by a similar argument, $x \in W_2$,
which is a contradiction.

\item{} backward direction

Assume that $W_1,W_2 \subseteq V$ are subspaces of $V$
such that $W_1 \subseteq W_2$ or $W_2 \subseteq W_1$.
We want to show that $W_1 \cup W_2$ is also a subspace of $V$.

We have two cases: first, $W_1 \subseteq W_2$;
and second, $W_2 \subseteq W_1$.

In the first case, $W_1 \cup W_2 = W_2$,
which we assumed is a subspace of $V$.

In the second case, $W_1 \cup W_2 = W_1$,
which we assumed is a subspace of $V$.
\bye
