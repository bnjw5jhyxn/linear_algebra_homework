\def\twomatrix#1#2#3#4{\left( {#1 \atop #2} {#3 \atop #4} \right)}
\centerline{Steve Hsu\hfill homework 9}
\item{5.2.2.} b. diagonalizable

$D = \twomatrix {-2} 0 0 4$ and $Q = \twomatrix 1 {-1} 1 1$
\medskip
\item{} d. diagonalizable

$D = \pmatrix{ -1 & 0 & 0\cr 0 & 3 & 0\cr 0 & 0 & 3\cr}$
and $Q = \pmatrix{ 2 & 1 & 0\cr 4 & 1 & 0\cr 3 & 0 & 1\cr }$
\medskip
\item{} f. not diagonalizable
\bigskip
\item{5.2.12.} a.

We will show that a vector $v \in V$
is in the eigenspace of $T$ corresponding to $\lambda$ if and only if
$v$ is in the eigenspace of $T^{-1}$ corresponding to $\lambda ^{-1}$.
By definition, $v$ is in the eigenspace of $T$ corresponding to $\lambda$
if and only if $Tv = \lambda v$.
If $Tv = \lambda v$, then applying $T^{-1}$ to both sides
and multiplying by $\lambda ^{-1}$,
we have that $\lambda ^{-1} v = T^{-1} v$.
Similarly, if $T^{-1} v = \lambda ^{-1} v$, then $\lambda v = Tv$.
By definition, $T^{-1} v = \lambda ^{-1} v$ if and only if
$v$ is in the eigenspace of $T^{-1}$ corresponding to $\lambda ^{-1}$.
\medskip
\item{} b.

Assume that $T$ is diagonalizable.
Then there is a basis $\beta = \{v_1, \ldots, v_n\}$ for $V$
and scalars $\lambda _1, \ldots, \lambda _n$ such that
$T v_i = \lambda _i$ for $1 \le i \le n$.
By part (a), we have that $T^{-1} v_i = \lambda _i ^{-1}$
so we can express $T^{-1}$ in diagonal form.
\bigskip
\item{5.4.2.} a. $T$-invariant

If $f(x) \in W$, then $f$ has degree at most $2$,
so $f'$ has degree at most $1$.
\medskip
\item{} b. not $T$-invariant

Notice that $f(x) = x^2$ is in $W$, but $x f(x) = x^3$ is not.
\medskip
\item{} c. $T$-invariant

If $(t, t, t) \in W$, then $T(t, t, t) = (3t, 3t, 3t) \in W$.
\medskip
\item{} d. $T$-invariant

Since $\int _0 ^1 f(x) dx$ is a constant, $Tf(t) = b$ for some $b$.
\medskip
\item{} e. $T$-invariant

If $\twomatrix a b b a \in W$, then
$T \twomatrix a b b a = \twomatrix b a a b \in W$.
\bigskip
\item{5.4.3.} a.

We will first show that $\{0\}$ is $T$-invariant.
Let $v \in \{0\}$.
We know that $v = 0$.
By linearity, $Tv = T0 = 0$, so $Tv \in \{0\}$.

We will now show that $V$ is $T$-invariant.
Let $v \in V$.
Since $T$ is a linear operator, $Tv \in V$.
\medskip
\item{} b.

We will first show that $N(T)$ is $T$-invariant.
Let $v \in N(T)$.
By definition, $Tv = 0$ and $0 \in N(T)$ since $N(T)$ is a subspace of $V$,
so $Tv \in N(T)$.

We will now show that $R(T)$ is $T$-invariant.
Let $v \in R(T)$.
Since $v \in V$, by definition, $Tv \in R(T)$.
\medskip
\item{} c.

Let $v \in E_\lambda$ for some eigenvalue $\lambda$ of $T$.
By definition, $Tv = \lambda v$.
Since $E_\lambda$ is a subspace of $V$,
it is closed under scalar multiplication,
so $\lambda v = Tv \in E_\lambda$.
\bigskip
\item{5.4.6.} d.

$\beta = \hbox{span} \left\{ \twomatrix 0 1 1 0, \twomatrix 1 2 1 2 \right\}$
\bigskip
\item{5.4.9.} d.

$z = \twomatrix 0 1 1 0$, $Tz = \twomatrix 1 2 1 2$, and $T^2 z = 3 6 3 6$,
so $-3Tz + T^2 z = 0$ and $f(t) = (-1)^2 (-3t + t^2) = t^2 - 3t$.

$f(t) = \det [T]_\beta = \det \twomatrix 0 1 0 3 = -t(3 - t) = t^2 - 3t$
\bigskip
\item{5.4.10.} d.

Let $\beta = \left\{ \twomatrix 1 0 0 0, \twomatrix 0 1 0 0,
\twomatrix 0 0 1 0, \twomatrix 0 0 0 1 \right\}$.
Then
$$[T]_\beta = \pmatrix{
1 & 1 & 0 & 0\cr
2 & 2 & 0 & 0\cr
0 & 0 & 1 & 1\cr
0 & 0 & 2 & 2\cr
}$$
so $f(t) = \det [T]_\beta - t I = (\det \twomatrix {1 - t} 2 1 {2 - t})^2 =
((1 - t)(2 - t) - 2)^2 = (t^2 - 3t)^2$.
Notice that $(t^2 - 3t) \vert (t^2 - 3t)^2$.
\bigskip
\item{5.4.19.}

For the base case, we have that $A_1 = (-a_0)$.
Clearly, the characteristic polynomial of $A_1$ is
$f_1(t) = -a_0 - t = (-1)^1 (a_0 + t^1)$.
For the inductive step, assume that
the characteristic polynomial of $A_k$ is
$f_k(t) = (-1)^k (a_0 + a_1 t + \cdots + a_{k-1} t^{k-1} + t^k)$.
Let $B_k = A_k - tI$.
We have that
$$f_{k+1}(t) = \det B_k = \det \pmatrix{
-t & 0 & \cdots & 0 & -a_0\cr
1 & -t & \cdots & 0 & -a_1\cr
0 & 1 & \cdots & 0 & -a_2\cr
\vdots & \vdots & & \vdots & \vdots\cr
0 & 0 & \cdots & -t & -a_{k-1}\cr
0 & 0 & \cdots & 1 & -t - a_k\cr
}$$
so $f_{k+1}(t) = -t \det B_{11} + (-1)^k (-a_0) B_{1(k+1)}$.
Notice that $A_{11}$ has the same structure as $B_k$,
so by the induction hypothesis, we have that
$\det B_{11} = (-1)^k (a_1 + a_2 t + \cdots + a_k t^{k-1} + t^k)$.
Additionally, $B_{1(k+1)}$ is an upper triangular matrix with only $1$s
on the diagonal, so $\det B_{1(k+1)} = 1$.
Substituting, we have that
$f_{k+1}(t) = -t (-1)^k (a_1 + a_2 t + \cdots + a_k t^{k-1} + t^k) + (-1)^k (-a_0) =
(-1)^{k+1} (a_1 t + a_2 t^2 + \cdots + a_k t^k + t^{k+1}) + (-1)^{k+1} a_0 =
(-1)^{k+1} (a_0 + a_1 t + \cdots + a_k t_k + t_{k+1})$, as desired.
\bye
