\def\textspan{\hbox{span}}
\def\rank{\hbox{rank}}
\def\nullity{\hbox{nullity}}
\centerline{Steve Hsu\hfill homework 9}
\item{7.1.2.} d.

$$K_2 = \textspan \left\{  \pmatrix{ 1\cr 0\cr 0\cr 0\cr }, \pmatrix{ 0\cr 1\cr 0\cr -1\cr } \right\} \qquad
K_3 = \textspan \left\{ \pmatrix{ 1\cr 1\cr 1\cr 0\cr }, \pmatrix{ 0\cr 0\cr 0\cr 1\cr } \right\} \qquad
J = \pmatrix{
2 & 1 & 0 & 0\cr
0 & 2 & 0 & 0\cr
0 & 0 & 3 & 0\cr
0 & 0 & 0 & 3\cr
}$$
\bigskip
\item{7.1.3.} a.

$$K_2 = \textspan \left\{ \pmatrix{ -1\cr 0\cr 0\cr }, \pmatrix{ 0\cr 1\cr 0\cr },
\pmatrix{ 0\cr 0\cr -1/2\cr } \right\} \qquad
J = \pmatrix{
2 & 1 & 0\cr
0 & 2 & 1\cr
0 & 0 & 2\cr
}$$
\bigskip
\item{7.2.3.} a.

$p_T(t) = (2 - t)^5 (3 - t)^2$
\medskip
\item{} b.

for $\lambda = 2$:

$\matrix{
\cdot & \cdot\cr
\cdot & \cdot\cr
\cdot & \cr
}$

for $\lambda = 3$:

$\matrix{ \cdot & \cdot\cr }$
\medskip
\item{} c.

The Jordan blocks for $K_3$ are $1 \times 1$, so $K_3 = E_3$.
\medskip
\item{} d.

For $K_2$, the largest Jordan block is $3 \times 3$,
so $K_2 = N((T - 2I)^3)$.
For $K_3$, the largest Jordan block is $1 \times 1$,
so $K_3 = N(T - 3I)$.
\medskip
\item{} e.

for $U_1 = T - 2I$ restricted to $K_2$:
\smallskip
\itemitem{i.} $\rank U_1 = 3$
\itemitem{ii.} $\rank U_1^2 = 1$
\itemitem{iii.} $\nullity U_1 = 2$
\itemitem{iv.} $\nullity U_1^2 = 4$
\smallskip
for $U_2 = T - 3I$ restricted to $K_3$:
\itemitem{i.} $\rank U_2 = 2$
\itemitem{ii.} $\rank U_2^2 = 2$
\itemitem{iii.} $\nullity U_2 = 0$
\itemitem{iv.} $\nullity U_2^2 = 0$
\bigskip
\item{7.2.12.}

Let $A$ be an $n \times n$ upper triangular matrix
with each diagonal entry equal to $0$.
We will show that $A^n = O$.

\proclaim Lemma. If $v \in F^n$ has only zeroes after the $k+1$th entry,
then $Av$ has only zeroes after the $k$th entry.

Using the row-column rule to multiply $A$ and $v$,
it follows that when computing the $\ell$th entry of $Av$, where $\ell \ge k$,
since $A$ is upper triangular, we have that $a_{\ell j} = 0$ for $j < \ell$
and $a_{\ell \ell} = 0$ since $A$ is zero on the diagonal.
Additionally, by our assumption, $v_j = 0$ for $j > \ell \ge k$,
so the $\ell$th entry of $Av$ is $0$, as desired.
\medskip
Notice that the bottom row of $A$ has $0$ in every entry.
For each column $v_i$ of $A$, we therefore have that $v_i$
has only zeroes after the $n-1$th entry.
By the lemma, $Av_i$ has only zeroes after the $n-2$th entry,
$A^2 v_i$ has only zeroes after the $n-3$th entry, and so on.
Inductively, $A^{n-1} v_i$ has only zeroes after the $0$th entry,
than is, $A^{n-1} v_i = 0$.
Since $A = (v_1 v_2 \cdots v_n)$, it follows that
$A^{n-1} A = A^n = O$, as desired.
\bye
