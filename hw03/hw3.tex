\def\zero{{\bf 0}}
\def\real{{\bf R}}
\centerline{Steve Hsu\hfill homework 3}
\item{1.6.} 15.

The basis is a set of matrices $A_{ij}$, where
$1 \le i,j \le n$ and $i,j$ are not both $n$.
Each $A_{ij}$ is a matrix with $1$ at entry $(i,j)$.
If $i = j$, then the matrix has a $-1$ at entry $(n,n)$.
It has $0$ at every other entry.

The vector space has dimension $n^2 - 1$.
\bigskip
\item{} 20.

We want to show that $S$ has at least $n$ elements.
Let $L$ be a basis for $V$.
By the definition of dimension, $L$ has exactly $n$ elements.
Since $L$ is linearly independent and $S$ generates $V$,
$S$ has at least $n$ elements by the Replacement Theorem.

We want to show that there is a subset of $S$ that is a basis for $V$.
We start with the empty set.
As long as the set has less than $n$ elements, we take an element of $S$
outside the span of this set and add it.
We will not stop before the set reaches $n$ elements.
If we do, then $S$, and consequently $V$, will be in the span of
fewer than $n$ vectors, violating the condition that $S$ has dimension $n$.
When we reach $n$ vectors, $S$ (and therefore $V$)
will be in the span of the set.
If it isn't, we will be able to find more than $n$ linearly independent
vectors in $V$, violating the condition that $S$ has dimension $n$.
\bigskip
\item{} 26.

$f$ must be of the form $f(x) = (x - a)g(x)$,
where $g \in P_{n - 1}(\real)$.
This subspace, like $P_{n - 1}(\real)$, has dimension $n$.
\bigskip
\item{2.1.} 2.

We want to show that $T(u + cv) = Tu + cTv$.
Let $u = (u_1, u_2, u_3)$ and $v = (v_1, v_2, v_3)$.
We have $T(u + cv) = T(u_1 + cv_1, u_2 + cv_2, u_3 + cv_3) =
(u_1 + cv_1 - u_2 - cv_2, 2u_3 + 2cv_3) =
(u_1 - u_2, 2u_3) + c(v_1 - v_2, 2v_3) = Tu + cTv$, as desired.

The set $\{(1,-1,0)\}$ is a basis for $N(T)$ and
$\{(1,0), (0,1)\}$ is a basis for $R(T)$.
The nullity of $T$ is therefore $1$, and
the rank of $T$ is $2$.
Their sum is $1 + 2 = 3$, which is indeed the dimension of the domain of $T$.

Since the set $\{T(1,0,0), T(0,0,1)\}$ generates $\real^2$, $T$ is onto.
\bigskip
\item{} 3.

We want to show that $T(u + cv) = Tu + cTv$.
Let $u = (u_1, u_2)$ and $v = (v_1, v_2)$.
We have $T(u + cv) = T(u_1 + cv_1, u_2 + cv_2) =
(u_1 + cv_1 + u_2 + cv_2, 0, 2u_1 + 2cv_1 - u_2 - cv_2) =
(u_1 + u_2, 0, 2u_1 - u_2) + c(v_1 + v_2, 0, 2v_1 - v_2) =
Tu + cTv$, as desired.

The empty set is a basis for $N(T)$ and
$\{(1,0,0),(0,0,1)\}$ is a basis for $R(T)$.
The nullity of $T$ is $0$ and the rank of $T$ is $2$.
Their sum is $0 + 2 = 2$, which is the dimension of $T$.

Since the nullity of $T$ is $0$, $T$ is one-to-one.
\bigskip
\item{} 9.
\itemitem{a.} $T(u + v) \ne Tu + Tv$

Let $u = (u_1, u_2)$ and $v = (v_1, v_2)$.
Then $T(u + v) = (1, u_2 + v_2)$ and
$Tu + Tv = (2, u_2 + v_2)$.
\medskip
\itemitem{b.} $T(cv) \ne cTv$

Let $v = (v_1, v_2)$.
Then $T(cv) = (cv_1, c^2 v_2^2)$ and
$cTv = (cv_1, cv_2^2)$.
\medskip
\itemitem{c.} $T(cv) \ne cTv$

Let $c = 2$ and $v = ({\pi \over 2}, 0)$.
Then $T(cv) = (0,0)$ and $cTv = (2,0)$.
\medskip
\itemitem{d.} $T(u + v) \ne Tu + Tv$

Let $u = (u_1, u_2)$ and $v = (-u_1, v_2)$.
Then $T(u + v) = (0, u_2 + v_2)$ and
$Tu + Tv = (2u_1, u_2 + v_2)$.
\bigskip
\item{} 13.

We will prove the claim by contraposition.
In other words, we will show that if $S$ is linearly dependent,
then $\{Tv_1, \cdots, Tv_k\}$ will be linearly dependent
and therefore cannot equal $\{w_1, \cdots, w_k\}$.

Assume that $S$ is linearly dependent.
Then there are constants $c_1, \cdots, c_k$, not all zero,
such that $\sum _{i = 1} ^k c_i v_i = \zero$.
Since $T$ is linear,
$\zero = T(\sum _{i = 1} ^k c_i v_i) =
\sum _{i = 1} ^k c_i Tv_i$.
The set $\{Tv_1, \cdots, Tv_k\}$ is therefore linearly dependent
and cannot equal $\{w_1, \cdots, w_k\}$.
\bigskip
\item{} 15.

To show that $T$ is linear,
we will show that $T(f + cg) = Tf + cTg$.
By the properties of integrals,
$T(f + cg) (x) =
\int _0 ^x (f + cg) (t) dt =
\int _0 ^x f(t) dt + \int _0 ^x cg(t) dt =
Tf(x) + c \int _0 ^x g(t) dt =
Tf(x) + cTg(x) = (Tf + cTg) (x)$, as desired.

We will show that $T$ is one-to-one.
It is sufficient to show that the $N(T)$ is a set containing
only the zero polynomial.
In other words, we will show that
if $Tf$ is the zero polynomial, then $f$ is the zero polynomial.
Assume that $Tf$ is the zero polynomial.
In other words, $\int _0 ^x f(t) dt = 0$ for all real numbers $x$.
The only polynomial satisfying this property is the zero polynomial,
so $f$ must be the zero polynomial, as desired.

To show that $T$ is not onto, we must find a polynomial $f$
such that $Tg \ne f$ for all polynomials $g$.
The polynomial $f(x) = 1$ for all real numbers $x$
is such a polynomial.
\bigskip
\item{} 17.
\itemitem{a.}

Let $\beta = \{v_1, \cdots, v_k\}$ be a basis for $V$.
Since $S = \{Tv_1, \cdots, Tv_k\}$
has fewer elements than a basis for $W$, it cannot generate $W$.
Any element of the range of $T$, however,
can be written as a linear combination of elements of $S$,
so $T$ is not onto.
\medskip
\itemitem{b.}

Let $\beta = \{v_1 \cdots, v_k\}$ be a basis for $V$.
Since $S = \{Tv_1, \cdots, Tv_k\}$
has more elements than a basis for $W$, it must be linearly dependent.
Then there are constants $c_1, \cdots, c_k$, not all zero,
such that $\sum _{i = 1} ^k c_i Tv_i = \zero$.
Since $T$ is linear, we have
$T(\sum _{i = 1} ^k c_i v_i) = \zero$.
Since $\beta$ is linearly independent, the sum cannot be $\zero$,
so we have a nonzero vector mapped to $\zero$ by $T$.
$T$ is therefore not one-to-one.
\bigskip
\item{} 20.

We will first show that $T(V_1)$ is a subspace of $W$.
In other words, we will show that $T(V_1)$
contains $\zero$, is closed under addition, and
is closed under multiplication.
Since $V_1$ contains $\zero$ and $T\zero = \zero$,
$T(V_1)$ must also contain $\zero$.
Let $c$ be an arbitrary element of $F$ and
let $u$ and $v$ be arbitrary elements of $V_1$.
Then $u + cv \in V_1$ since $V_1$ is a subspace of $V$,
and $Tu, T(cv) \in T(V_1)$.
Therefore, since $T$ is linear, $Tu + cTv = T(u + cv) \in T(V_1)$,
so $T(V_1)$ is closed under addition and scalar multiplication.

Let $V_2 = \{x \in V : Tx \in W_1\}$.
We want to show that $V_2$ contains $\zero$
and is closed under addition and scalar multiplication.
Since $W_1$ contains $\zero$ and $T\zero = \zero$,
$V_2$ must also contain $\zero$.
Let $c$ be an arbitrary element of $F$ and
let $u$ and $v$ be arbitrary elements of $V_2$.
Since $W_1$ is a subspace of $W$, $Tu + cTv \in W_1$.
Since $T$ is linear, $Tu + cTv = T(u + cv)$, so $u + cv \in V_2$.
Therefore, $V_2$ is closed under addition and scalar multiplication.
\bye
